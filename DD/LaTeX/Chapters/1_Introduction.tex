\section{Purpose}
\label{sec:purpose}%
The purpose of this document is to present a detailed description of Students\&Company.
It is addressed to the developers who have to implement the requirements and could be used as an agreement between the customer and the contractors.\\ 
The document is also intended to provide the customer with a clear and unambiguous description of the system's functionalities and constraints, allowing the customer to validate the requirements and to verify if the system meets the expectations.

\section{Scope}
\label{sec:scope}%
Students\&Companies (S\&C) is a platform designed to simplify and optimize the matching process between university students looking for internship opportunities and companies offering them. The system analyzes students' profiles, CVs, and indicated preferences, matching them with internship offers posted by companies, which include details on required skills, technologies used, and proposed conditions. Using advanced recommendation algorithms, S\&C suggests suitable opportunities to students and notifies companies of candidates who best meet their needs. The platform also supports the entire selection cycle, from application and interview management to feedback collection, ensuring a structured and transparent experience for all users involved, including universities that monitor the progress of internships and address any issues.



\section{Definition, Acronyms, Abbreviations}
\label{sec:definition_acronyms_abbreviations}%

\begin{table}[H]
    \begin{center}
        \begin{tabular}{ |l|l| }
            \hline
            \textbf{Acronyms} & \textbf{Definition}                              \\
            \hline
            DD             & Design Document                      \\
            \hline            
            RASD             & Requirements Analysis \& Specification Document     \\   
            \hline
            ST              & Student                         \\
            \hline
            ED              & Educator                         \\
            \hline
            STG             & Student Group                    \\
            \hline
            CKB             & CodeKataBattle                   \\
            \hline
            GH              & GitHub                           \\
            \hline
            User            & All STs and EDs                           \\
            \hline
            API             & Application Programming Interface       \\
            \hline
            RX              & Requirement X                           \\
            \hline
            CMP            & Component                           \\
            \hline
         \end{tabular}
        \caption{Acronyms used in the document.}
        \label{tab:acronyms}%
    \end{center}
\end{table}

\section{Revision History}
\label{sec:revision_history}%
\textbf{Version 1.0} - 07/01/2024

\section{Reference Documents}
\label{sec:reference_documents}%

\begin{itemize}
    \item Specification Document Assignment
\end{itemize}

\section{Document Structure}
\label{sec:doc_structure}%
The document is structured in seven sections, as described below.

\textbf{Introduction}. In the first section, the chapter elucidates the significance of the Design 
Document, providing comprehensive definitions and explanations of acronyms and abbreviations. Additionally, it recalled the scope of the CodeKateBattle system.

\textbf{Architectural Design}. The second section shows the main components of the system and their relationships. This section also focuses on design choices and architectural styles, patterns and paradigms.

\textbf{User Interface Design}. The next section, the third, describes the user interface of the system, providing mockups and explanations of the main pages.

\textbf{Requirements Traceability}. The fourth section describes the requirements of the system, showing how they are satisfied by the design choices.

\textbf{Implementation, Integration and test Plan}. This fifth part provides an overview of the implementation of the various components of the system, it also shows how they are integrated and it gives a plan for testing them all.

\textbf{Effort Spent}. In the sixth section are included information about the number of hours each group member has worked for this document.

\textbf{References}. The last section contains the list of the documents used to redact this Design Document.
 
